\def\UseMinted{}
\documentclass[../main.tex]{subfiles}
\begin{document}

\section{Napotkane problemy}
\subsection{Wczesne podejście z LD\_PRELOAD}
Pierwszym pomysłem na nadpisanie funkcji \mintinline{c}{__sanitizer_on_print()}
było napisanie dynamicznej biblioteki definiującej tę funkcje, a następnie
wywołanie testowanego programu ze zmienną
\mintinline{sh}{LD_PRELOAD=libsan-overlay.so}.

Ta metoda działa i jej największą zaletą jest to, że z
\mintinline{sh}{LD_PRELOAD}, tak długo jak mamy program zlinkowany dynamicznie
z \mintinline{sh}{libasan.so}, nie potrzebujemy dostępu ani do żadnych źródeł
ani plików obiektowych. Niestety, są też wady, które dyskwalifikują to
rozwiązanie:

\begin{enumerate}
	\item Metoda jest uzależniona od mechanizmu systemu operacyjnego i nie
		działałaby na bare metal.
	\item Konieczne jest przeniesienie na platformę
		\mintinline{sh}{libsan-overlay.so} wraz z programem testowym oraz
		ustawienie \mintinline{sh}{LD_PRELOAD}, zatem potencjalnie musimy
		ingerować w skrypty uruchamiające program.
	\item Metoda polega na dostępności \mintinline{sh}{libasan.so} na
		platformie. Na okrojonych instalacjach Linux, ta biblioteka
		najprawdopodobniej nie będzie istnieć, zatem pojawia się konieczność
		dostarczenia jej wraz z programem. Jest to szczególnie uciążliwe, jeśli
		platforma stoi na nietypowej architekturze, bo trzeba znaleźć
		odpowiednią wersję \mintinline{sh}{libasan.so} i mieć nadzieję, że nie
		będzie żadnych niekompatybilności między nią a dynamicznie zlinkowanym
		programem testowym (co nie jest oczywiste np. w przypadku różnic w
		wersjach kompilatora).
\end{enumerate}

\subsection{Nadpisanie funkcji ze statycznie linkowanej biblioteki}
Zanim wpadliśmy na pomysł połączenia dwóch plików obiektowych w jeden przed
linkowaniem, rozważaliśmy sposoby na nadpisanie funkcji \textbf{po} linkowaniu.

W grę wchodzą dwa fundamentalne podejścia: modyfikacja funkcji wewnątrz pliku
binarnego (przyjmijmy ELF), lub nadpisanie adresu funkcji w trakcie wykonania
programu (tzw. "trampolina").

\subsubsection{Modyfikacja ELF}
W teorii, posiadając skompilowany plik binarny który nie został strip'owany,
powinno być możliwe znalezienie miejsca, gdzie zapisane są instrukcje \\
\mintinline{c}{__sanitizer_on_print()} i podmiana ich na własną definicję. W
praktyce, nie znaleźliśmy narzędzi, które w łatwy sposób umożliwiałyby
zrobienie czegoś takiego. Co więcej, nawet gdyby takie narzędzie istniało,
zapewne byłoby bardzo specyficzne dla danej architektury lub formatu ELF i nie
byłoby dostatecznie uniwersalne, aby umożliwić testowanie na wielu różnych
platformach.

\subsubsection{Trampolina}
Nie ingerując w plik binarny, można rozważyć podejście dynamiczne: załadowanie
do pamięci własnej implementacji \mintinline{c}{__sanitizer_on_print()}, a
następnie sprytna podmiana wszystkich skoków pamięci do oryginalnej wersji na
nasz własny adres.

Takie metody się stosuje i można znaleźć nawet przykłady w internecie.
Przeważnie polegają na napisaniu biblioteki dynamicznej z funkcją inicjalizacji
(w GCC \mintinline{c}{__attribute__((constructor))}), która wykorzystuje
\mintinline{c}{dlopen()} i \mintinline{c}{dlsym()} aby znaleźć adres funkcji,
którą chcemy podmienić, a następnie wpisuje w odpowiednie miejsca (w pamięci)
instrukcje assemblera do skoku pod inny adres.

Problem tych metod jest taki, że są bardzo mało uniwersalne, wymagają pisania
kodu pod konkretną architekturę, przeważnie polegają na dynamicznym ładowaniu
bibliotek (a zatem niekompatybilne z bare metal) i nie należą do łatwych w
implementacji. Dodatkowo mogłyby być blokowane przez zabezpieczenia sprzętowe
typu NX bit.
\end{document}
