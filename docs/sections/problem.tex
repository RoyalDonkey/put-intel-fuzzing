\def\UseMinted{}
\documentclass[../main.tex]{subfiles}
\begin{document}

\section{Opis problemu}
\paragraph{Fuzzing}
Problem dotyczył fuzzingu, czyli metody testowania oprogramowania polegającej
na celowym generowaniu niepoprawnych, lub pół-poprawnych danych wejściowych
przez osobny program (tzw. fuzzer). Tak wygenerowane dane są następnie podawane
do testowanego programu, w celu sprowokowania nieprawidłowego działania.
Fuzzery są w zasadzie programami szukającym nowych danych do korpusu testowego.

Fuzzing jest gorącym tematem, który rozwija się prężnie od kilku dekad. Obecnie
większość projektów i firm dużej skali takich jak jądro Linux, Microsoft,
Google, etc. stosuje  fuzzing z wielkim powodzeniem, ponieważ te narzędzia
wykrywają błędy, które bardzo trudno jest zauważyć gołym okiem lub przy użyciu
tradycyjnych metod takich jak testy jednostkowe.

\paragraph{Sanitizery}
Aby fuzzing był skuteczny do wdrożenia na dużą skalę, potrzebne są narzędzia do
analizy poprawności stanu testowanego programu, aby umożliwić automatyczne
wykrycie, kiedy fuzzer znalazł podatność. Jedną z klas takich narzędzi są
sanitizery pamięci. Na potrzeby tego projektu istotny jest jeden konkretny
sanitizer, czyli ASAN.

ASAN monitoruje stan pamięci testowanego programu w trakcie jego wykonywania i
natychmiast zawiadamia o błędach (np. przepełnienie bufora), jeśli takowe
wystąpią. Taka analiza jest możliwa, ponieważ instrukcje ASANa są bezpośrednio
wkompilowane w testowany program (tak zwana “instrumentalizacja” programu).
Testowany program funkcjonalnie pozostaje niezmieniony, ale na każdym kroku ma
wbudowane dodatkowe sprawdzenia odczytów, zapisów, alokacji i zwalniania
pamięci.

ASAN jest w sporej części po prostu biblioteką napisaną w C (libasan.so -- na
systemach Linux). W kompilatorze GCC, po instrumentalizacji testowanego
programu, jest on zlinkowany dynamicznie z tą biblioteką. Zatem, aby uruchomić
testowany program, system operacyjny musi znaleźć tę zależność.

\paragraph{Problem}
Istotą naszego zadania było znalezienie metody na przeprowadzenie fuzzingu
programów na systemach wbudowanych bądź bare metal, wewnątrz emulatora Intel
Simics. Testowanie w takich warunkach jest z wielu powodów trudniejsze. W
naszej pracy pomijamy element fuzzera i skupiamy się na sanitizerze.

Na szczególną uwagę zasługuje bare metal, na którym jest zupełny brak systemu
operacyjnego, więc nie ma możliwości nawet podejrzenia komunikatów printowanych
przez ASANa. Aby umożliwić testowanie ASANem na bare metal, konieczne jest
utworzenie kanału komunikacji między nim a Simicsem, tj. chcemy, aby każda
wiadomość od ASANa, która klasycznie wylądowałaby na standardowym wyjściu
programu, była dostępna jako string w skrypcie Simicsowym. Brzmi to
abstrakcyjnie, bo takie jest. Zasadniczo, to jest największy problem projektu i
cała nasza praca głównie sprowadziła się do tego zadania.
\end{document}
