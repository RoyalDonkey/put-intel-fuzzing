\def\UseMinted{}
\documentclass[../main.tex]{subfiles}
\begin{document}
		
	\section{Kierunki Rozwoju}
Dalszą pracą, jaką można byłoby wykonać, można podzielić na dwa główne zadania.

		\subsection{Bare-metal}
 Jako pierwszy kierunek dalszego rozwoju projektu, należy wymienić zbudowanie programu i uruchomienie go bez systemu operacyjnego (czyli bare-metal). Wymagałoby to standardowych kroków dla takiej implementacji oprogramowania, a następnie, wykorzystując m.in . już wcześniej opisaną przez nas metodę kompilacji z sanitizerem programów statycznie, stworzenia pliku z kodem maszynowym, który mógłby wykonać procesor.



		\subsection{Fuzzing}
		Drugim kierunkiem niewątpliwie jest dodanie fuzzera do aplikacji, którą chcielibyśmy testować. Program ten powinien przyjmować jakieś wejście od użytkownika, a następnie przetwarzać je, aby możliwym było prześledzenie, czy program zachowuje się prawidłowo.


\end{document}